\documentclass[]{article}
\usepackage{lmodern}
\usepackage{amssymb,amsmath}
\usepackage{ifxetex,ifluatex}
\usepackage{fixltx2e} % provides \textsubscript
\ifnum 0\ifxetex 1\fi\ifluatex 1\fi=0 % if pdftex
  \usepackage[T1]{fontenc}
  \usepackage[utf8]{inputenc}
\else % if luatex or xelatex
  \ifxetex
    \usepackage{mathspec}
    \usepackage{xltxtra,xunicode}
  \else
    \usepackage{fontspec}
  \fi
  \defaultfontfeatures{Mapping=tex-text,Scale=MatchLowercase}
  \newcommand{\euro}{€}
\fi
% use upquote if available, for straight quotes in verbatim environments
\IfFileExists{upquote.sty}{\usepackage{upquote}}{}
% use microtype if available
\IfFileExists{microtype.sty}{%
\usepackage{microtype}
\UseMicrotypeSet[protrusion]{basicmath} % disable protrusion for tt fonts
}{}
\usepackage{color}
\usepackage{fancyvrb}
\newcommand{\VerbBar}{|}
\newcommand{\VERB}{\Verb[commandchars=\\\{\}]}
\DefineVerbatimEnvironment{Highlighting}{Verbatim}{commandchars=\\\{\}}
% Add ',fontsize=\small' for more characters per line
\usepackage{framed}
\definecolor{shadecolor}{RGB}{248,248,248}
\newenvironment{Shaded}{\begin{snugshade}}{\end{snugshade}}
\newcommand{\KeywordTok}[1]{\textcolor[rgb]{0.13,0.29,0.53}{\textbf{{#1}}}}
\newcommand{\DataTypeTok}[1]{\textcolor[rgb]{0.13,0.29,0.53}{{#1}}}
\newcommand{\DecValTok}[1]{\textcolor[rgb]{0.00,0.00,0.81}{{#1}}}
\newcommand{\BaseNTok}[1]{\textcolor[rgb]{0.00,0.00,0.81}{{#1}}}
\newcommand{\FloatTok}[1]{\textcolor[rgb]{0.00,0.00,0.81}{{#1}}}
\newcommand{\CharTok}[1]{\textcolor[rgb]{0.31,0.60,0.02}{{#1}}}
\newcommand{\StringTok}[1]{\textcolor[rgb]{0.31,0.60,0.02}{{#1}}}
\newcommand{\CommentTok}[1]{\textcolor[rgb]{0.56,0.35,0.01}{\textit{{#1}}}}
\newcommand{\OtherTok}[1]{\textcolor[rgb]{0.56,0.35,0.01}{{#1}}}
\newcommand{\AlertTok}[1]{\textcolor[rgb]{0.94,0.16,0.16}{{#1}}}
\newcommand{\FunctionTok}[1]{\textcolor[rgb]{0.00,0.00,0.00}{{#1}}}
\newcommand{\RegionMarkerTok}[1]{{#1}}
\newcommand{\ErrorTok}[1]{\textbf{{#1}}}
\newcommand{\NormalTok}[1]{{#1}}


\usepackage{longtable,booktabs}
\usepackage{graphicx}
\makeatletter
\def\maxwidth{\ifdim\Gin@nat@width>\linewidth\linewidth\else\Gin@nat@width\fi}
\def\maxheight{\ifdim\Gin@nat@height>\textheight\textheight\else\Gin@nat@height\fi}
\makeatother
% Scale images if necessary, so that they will not overflow the page
% margins by default, and it is still possible to overwrite the defaults
% using explicit options in \includegraphics[width, height, ...]{}
\setkeys{Gin}{width=\maxwidth,height=\maxheight,keepaspectratio}
\ifxetex
  \usepackage[setpagesize=false, % page size defined by xetex
              unicode=false, % unicode breaks when used with xetex
              xetex]{hyperref}
\else
  \usepackage[unicode=true]{hyperref}
\fi
\hypersetup{breaklinks=true,
            bookmarks=true,
            pdfauthor={},
            pdftitle={An awesome paper on an amazing topic},
            colorlinks=true,
            citecolor=blue,
            urlcolor=blue,
            linkcolor=magenta,
            pdfborder={0 0 0}}
\urlstyle{same}  % don't use monospace font for urls
\setlength{\parindent}{0pt}
\setlength{\parskip}{6pt plus 2pt minus 1pt}
\setlength{\emergencystretch}{3em}  % prevent overfull lines
\setcounter{secnumdepth}{0}

% a0poster {
% }



% title {
  \title{An awesome paper on an amazing topic}
      % }


% authors {
\author{
                    Author1  (University of A1) 
                       \and
                    Author2  (A2 Institute of Technology) 
                      }
% }

\date{}


% letter {
% }

\begin{document}

% a0poster {
% }

\maketitle

% a0poster {
% }

\begin{abstract}
There is the need for an awesome system, so we built one.
\end{abstract}




% a0poster {
% }


% letter {
% }


\section{Introduction}\label{introduction}

The intro goes here. We can cite existing work (Lamport 1978) and some
more (Gray et al. 1976 ; Lampson 1996 ; Stonebraker and Hellerstein
1988).

The article can also include \href{http://wikipedia.org}{links to
pages}. We can also refer to other sections, for example,
\hyperref[bg]{see~}.

\hyperdef{}{bg}{\section{Background}\label{bg}}

Then some background so that people can understand

\section{Our System}\label{our-system}

The architecture of the system can be seen below (Figure 1).

\begin{figure}[htbp]
\centering
\includegraphics{figures/square.png}
\caption{Our system is fairly simple. It consists of a single rectangle}
\end{figure}

\subsection{The high-level features}\label{the-high-level-features}

The article can also tables. And a table listing some things:

\begin{longtable}[c]{@{}clrl@{}}
\caption{Here's the caption. It, too, may span multiple
lines.}\tabularnewline
\toprule
\begin{minipage}[b]{0.15\columnwidth}\centering\strut
Centered Header
\strut\end{minipage} &
\begin{minipage}[b]{0.10\columnwidth}\raggedright\strut
Default Aligned
\strut\end{minipage} &
\begin{minipage}[b]{0.20\columnwidth}\raggedleft\strut
Right Aligned
\strut\end{minipage} &
\begin{minipage}[b]{0.31\columnwidth}\raggedright\strut
Left Aligned
\strut\end{minipage}\tabularnewline
\midrule
\endfirsthead
\toprule
\begin{minipage}[b]{0.15\columnwidth}\centering\strut
Centered Header
\strut\end{minipage} &
\begin{minipage}[b]{0.10\columnwidth}\raggedright\strut
Default Aligned
\strut\end{minipage} &
\begin{minipage}[b]{0.20\columnwidth}\raggedleft\strut
Right Aligned
\strut\end{minipage} &
\begin{minipage}[b]{0.31\columnwidth}\raggedright\strut
Left Aligned
\strut\end{minipage}\tabularnewline
\midrule
\endhead
\begin{minipage}[t]{0.15\columnwidth}\centering\strut
First
\strut\end{minipage} &
\begin{minipage}[t]{0.10\columnwidth}\raggedright\strut
row
\strut\end{minipage} &
\begin{minipage}[t]{0.20\columnwidth}\raggedleft\strut
12.0
\strut\end{minipage} &
\begin{minipage}[t]{0.31\columnwidth}\raggedright\strut
Example of a row that spans multiple lines.
\strut\end{minipage}\tabularnewline
\begin{minipage}[t]{0.15\columnwidth}\centering\strut
Second
\strut\end{minipage} &
\begin{minipage}[t]{0.10\columnwidth}\raggedright\strut
row
\strut\end{minipage} &
\begin{minipage}[t]{0.20\columnwidth}\raggedleft\strut
5.0
\strut\end{minipage} &
\begin{minipage}[t]{0.31\columnwidth}\raggedright\strut
Here's another one. Note the blank line between rows.
\strut\end{minipage}\tabularnewline
\bottomrule
\end{longtable}

\subsection{The Details}\label{the-details}

Then some details. With some cpp code:

\begin{Shaded}
\begin{Highlighting}[]
\OtherTok{#include <iostream.h>}

\NormalTok{main()}
\NormalTok{\{}
    \NormalTok{cout << }\StringTok{"Hello World!"}\NormalTok{;}
    \KeywordTok{return} \DecValTok{0}\NormalTok{;}
\NormalTok{\}}
\end{Highlighting}
\end{Shaded}

We can also use footnotes\footnote{This footnote is for illustration
  purposes only, don't take it too seriously.}.

\section{Bibliography}\label{bibliography}

\noindent
\vspace{-2em} \setlength{\parindent}{-0.26in}
\setlength{\leftskip}{0.2in} \setlength{\parskip}{8pt}

Gray, Jim, Raymond A. Lorie, Gianfranco R. Putzolu, and Irving L.
Traiger. 1976. ``Granularity of Locks and Degrees of Consistency in a
Shared Data Base.'' In \emph{IFIP Working Conference on Modelling in
Data Base Management Systems}, 365--94.

Lamport, Leslie. 1978. ``Time, Clocks, and the Ordering of Events in a
Distributed System.'' \emph{Commun. ACM} 21 (7): 558--65.
doi:\href{http://dx.doi.org/10.1145/359545.359563}{10.1145/359545.359563}.

Lampson, Butler W. 1996. ``How to Build a Highly Available System Using
Consensus.'' In \emph{Distributed Algorithms}, edited by Özalp Babaoğlu
and Keith Marzullo, 1--17. Lecture Notes in Computer Science 1151.
Springer Berlin Heidelberg.

Stonebraker, Michael, and Joseph M. Hellerstein. 1988. \emph{Readings in
Database Systems}. Vol. 1. The MIT Press.

% letter {
% }

% a0poster {
% }
\end{document}
